\section{Task 1}
Question 1
-	COVID and normal chest X-rays can have subtle visual differences, making it hard for a model to learn the differences between classes.
-	The dataset size is relatively small (only 1600 training images), which increases the risk of overfitting.
-	X-ray images often have variability in terms of brightness, contrast, and anatomical differences between individuals, which introduce noise
-	If the dataset is not perfectly balanced the model might become biased toward the majority class but fortunately, the plot shows that this is not the case here.
Question 2
Based on the pixel statistics and label distribution, the datasets appear to be fairly uniformly divided, though not perfectly. The average pixel values and standard deviations are similar across sets — training (mean ≈ 0.53, std ≈ 0.25), validation (mean ≈ 0.55, std ≈ 0.26), and test (mean ≈ 0.55, std ≈ 0.26) — indicating consistent image intensity. The class distribution also looks well balanced overall. The training set is evenly split, while the validation set shows a slight tilt toward NORMAL, and the test set has a few more COVID samples. These minor differences are not concerning. Overall, the datasets are reasonably well-balanced in terms of both pixel characteristics and label distribution.
